\documentclass{article}
\usepackage{amsmath, amssymb, verbatim, amsthm}
\newcommand{\contradiction}{\Rightarrow\!\Leftarrow}
% I need a title
\author{Minh Bui}
\title{Math 323 HW22}

\theoremstyle{claim}
\newtheorem{claim}{Claim}
\newtheorem{theorem}{Theorem}[section]
\newtheorem{corollary}{Corollary}[theorem]
\newtheorem{lemma}[theorem]{Lemma}
\theoremstyle{definition}
\newtheorem{definition}{Definition}
\begin{document}
% Generates the title
\maketitle
\begin{enumerate}
    \item[1.] Prove that the integers is countable.
        \begin{proof}
            To prove that the set of integers is countable, we need to find an injection (or one-to-one relationship) that maps $\mathbb{Z}$ to $\mathbb{N}$: $i: \mathbb{Z} \rightarrow \mathbb{N}$.\\
            Let
            \begin{equation*}
            i(x) = \left\{
                \begin{array}{rl}
                    2x + 1 & \text{ if } x \ge 0,\\
                    -2x & \text{ if } x < 0.
                \end{array} \right.
            \end{equation*}
            We need to show that $i(x)$ is an injection.\\
            Assume $x_1, x_2 \in \mathbb{Z}$ and $i(x_1) = i(x_2)$. We have 3 cases.
            \begin{enumerate}
                \item[1.] $2x_1 + 1 = 2x_2 + 1$\\
                    Assume that is the case. We have $2x_1  = 2x_2$ and $x_1 = x_2$.
                \item[2.] $-2x_1 = -2x_2$\\
                    Assume that is the case. We have $x_1 = x_2$.
                \item[3.] $2x_1 + 1 = -2x_2$\\
                    Assume BWOC that is the case. Then $x_1 \ge 0$ and $x_2 < 0$. From the equation, $x_1 = \frac{-2x_2 -1}{2}$. But we know $x_1, x_2 \in \mathbb{Z}$ and $\frac{-2x_2 - 1}{2} \notin \mathbb{Z}$. So this case can't happen.
            \end{enumerate}
            Thus, $i(x)$ is an injection. And so the set of integers $\mathbb{Z}$ is countable.
        \end{proof}
    \item[2.] Let $A$ and $B$ be countable sets. Prove that $A \cup B$ is countable.
    \begin{proof}
        Let $A$ and $B$ be countable sets. This respectively means,
        \begin{gather*}
            \text{ There is an injection } f: A \rightarrow \mathbb{N}\\
            \text{ There is an injection } g: B \rightarrow \mathbb{N}
        \end{gather*}
        We define $i_3: A \cup U \rightarrow \mathbb{N}$
        \begin{equation*}
            h(x) = \left\{
                \begin{array}{rl}
                    2(f(x)) \text{ if } x \in A\\
                    2(g(x)) + 1\text{ if } x \in B
                \end{array} \right.
        \end{equation*}
        Want to show: There is a function $h: A \cup B \rightarrow \mathbb{N}$ such that $h$ is an injection.\\
        Assume $x_1, x_2 \in A \cup B$ and $h(x_1) = h(x_2)$. We have 3 cases to consider.
        \begin{enumerate}
            \item[1.] $x_1, x_2 \in A$.\\
                Assume that is the case. Then we have $2(f(x_1)) = 2(f(x_2))$. So then $f(x_1) = f(x_2)$. Since $f$ is an injection, $x_1 = x_2$.
            \item[2.] $x_1, x_2 \in B$.\\
                Assume that is the case. Then we have $2(g(x_1)) + 1 = 2(g(x_2)) + 1$. So then $g(x_1) = g(x_2)$. Since $g$ is an injection, $x_1 = x_2$.
            \item[3.] $x_1 \in A$ and $x_2 \in B$.\\
                Assume BWOC that is the case. Then we have $2(f(x_1)) = 2(g(x)) + 1$. Since $f: A \rightarrow \mathbb{N}$ and so is $g$, $2(f(x_1))$ and $2(g(x)) + 1$ are in $\mathbb{R}$. So then $2(f(x_1))$ is even and $2(g(x)) + 1$ is odd and they are equal. This is a contradition, this case cannot happen.
        \end{enumerate}
        Thus, if $A$ and $B$ are countable sets, then $A \cup B$ is countable.
    \end{proof}
\item[3.] Let $f: \mathbb{R} \rightarrow \mathbb{R}$ and $g: \mathbb{R} \rightarrow \mathbb{R}$ be functions that are both continuous at $1$. Prove that $f + g$ is continuous at $1$.
        \begin{proof}
            Let $f: \mathbb{R} \rightarrow \mathbb{R}$ and $g: \mathbb{R} \rightarrow \mathbb{R}$ be functions that are both continuous at $1$. So
            \begin{gather*}
                \forall \epsilon > 0, \exists \delta_1 > 0 \text{ s.t } |x - 1| < \delta_1 \text{ implies } |f(x) - f(1)| < \epsilon\\
                \forall \epsilon > 0, \exists \delta_2 > 0 \text{ s.t } |x - 1| < \delta_2 \text{ implies } |g(x) - g(1)| < \epsilon
            \end{gather*}
            We want to show that $|f(x) + g(x) - f(1) - g(1)| < \epsilon$.
            Let $\epsilon > 0$. So $\frac{\epsilon}{2} > 0$. Assume there is $\delta_1 > 0$ so that $|x - 1| < \delta_1$. Let $x \in \mathbb{R}$. Assume there is $\delta_2 > 0$ so that $|x - 1| < \delta_2$. So then $|x - 1| < min\{\delta_1, \delta_2\}$. Since $|f(x) - f(1)| < \epsilon$ for all $\epsilon > 0$, $|f(x) - f(1)| < \frac{\epsilon}{2}$. Same thing happens to $|g(x) - g(1)|$. Then consider
            \begin{gather*}
                |f(x) - f(1)| < \frac{\epsilon}{2} \text{ and } |g(x) - g(1)| < \frac{\epsilon}{2}\\
                |f(x) - f(1)| + |g(x) - g(1)| < \epsilon\\
                |f(x) - f(1) + g(x) - g(1)| \le |f(x) - f(1)| + |g(x) - g(1)| < \epsilon\\
                |f(x) + g(x) - f(1) - g(1)| < \epsilon
            \end{gather*}
            Thus $f + g$ is continuous at $1$.
        \end{proof}
\end{enumerate}
\end{document}
