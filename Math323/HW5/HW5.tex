\documentclass{article}
\usepackage{amsmath, amssymb, verbatim, amsthm}
\newcommand{\contradiction}{\Rightarrow\!\Leftarrow}
% I need a title
\author{Minh Bui}
\title{Math 323 HW5}

\theoremstyle{claim}
\newtheorem{claim}{Claim}

\theoremstyle{definition}
\newtheorem{definition}{Definition}
\begin{document}
% Generates the title
\maketitle
\begin{enumerate}
    \item[Problem 3.2:] Give an example of a set of integers with at least one element but no minimum.
    \item[] \emph{Solution.} Let $A = \{k \in \mathbb{Z} \mid k < 0\}$ is a nonempty set of integers that does not have a minimum.
    \item[Problem 3.4:] Consider $A = \{m \in \mathbb{Z} \mid m > 17\}$.
        \begin{enumerate}
            \item Is 17 a lower bound of $A$?
            \item Is 12 a lower bound of $A$?
            \item Is 20 a lower bound of $A$?
            \item Is 17 a minimum of $A$?
            \item Is 12 a minimum of $A$?
            \item Is 20 a minimum of $A$?
            \item Is 18 a lower bound of $A$?
            \item Is 18 a minimum of $A$?
        \end{enumerate}
    \item[] \emph{Solution:} We review our definition of lower bound and minimum from Prof. Madden's book.
        \begin{definition}
            In an ordered number system the number $l$ is said to be a lower bound on a set $S$ when: if $s \in S$, then $l \le s$.
        \end{definition}
        \begin{definition}
            Let $S$ be a set of numbers. We say $m$ is a minimum of the set $S$ when
            \begin{enumerate}
                \item[1.] $m \in S$
                \item[2.] If $s \in S$, then $m \le s$
            \end{enumerate}
        \end{definition}
        \begin{enumerate}
            \item 17 is a lower bound of $A$. 
            \item 12 a lower bound of $A$.
            \item 20 is not a lower bound of $A$ since $20 \in A$, $18 \in A$ but $20 > 18$.
            \item 17 is not a minimum of $A$ since it's not in the set $A$.
            \item 12 is not a minimum of $A$ since it's not in the set $A$.
            \item 20 is not a minimum of $A$ since $18 \in A$ but $20 > 18$.
            \item 18 is a lower bound of $A$ since for $m \in \mathbb{Z}$ and $m > 17$, $m \ge 18$.
            \item 18 is a minimum of $A$ since $18 \in A$ and $\forall m \in A$, $m \ge 18$.
        \end{enumerate}

    \item[Problem 3.12:] Let $A$ and $B$ be sets of integers so that every element of $A$ is also an element of $B$. Let $r, s \in \mathbb{Z}$ with $r < s$. Are the following true or false?
        \begin{enumerate}
            \item If $r$ is a lower bound on $A$, then $s$ is a lower bound on $A$.
                \item[] False. If there is an integer $k$ such that $r < k < s$ and if $k \in A$ then $s$ is not a lower bound of $A$.
            \item If $s$ is a lower bound on $A$, then $r$ is a lower bound on $A$.
                \item[] True since $s$ is smaller than every elements in A and $r < s$.
            \item If $r$ is a lower bound on $A$, then $s$ is a lower bound on $B$.
                \item[] False. We know $A \subseteq B$. So there is a case that $A = B$. Assume set $A$ or $B$ has a minimum and assume such minimum is actually $r$. Assume $s \in B$. We know $s > r$. And so $s$ cannot be a lower bound on $B$.
            \item If $s$ is a lower bound on $A$, then $r$ is a lower bound on $B$.
                \item[] False. We know $A \subseteq B$, but what if $\exists k \in B$ and $k \notin A$ such that $k < r$?
            \item If $r$ is a lower bound on $B$, then $s$ is a lower bound on $A$.
                \item[] False. $s$ is not a lower bound on $A$ if $\exists a \in A$ such that $r < a < s$.
            \item If $s$ is a lower bound on $B$, then $r$ is a lower bound on $A$.
                \item[] True. 
        \end{enumerate}
    \item[Problem 3.13:] Let $A$ be a set of integers and $m \in \mathbb{Z}$. Define the terms:
        \begin{enumerate}
            \item $m$ is a maximum of $A$.
            \item $m$ is an upper bound of $A$.
        \end{enumerate}
    \item[] \emph{Solution.}
        \begin{definition}
            Let $A$ be a set of integers and $m \in \mathbb{Z}$. We say $m$ is a maximum of $A$ if and only if
            \begin{enumerate}
                \item[1.] $m \in A$
                \item[2.] If $a \in A$, then $a \le m$
            \end{enumerate}
        \end{definition}
        \begin{definition}
            Let $A$ be a set of integers and $m \in \mathbb{Z}$. We say $m$ is an upper bound of $A$ if and only if it satisfies this condition: if $a \in A$, then $m \ge a$.
        \end{definition}
    \item[Problem 3.11:] Let $a, b \in \mathbb{Z}$. Prove: If $a \ne b$, then $ab \ne 1$.
        \begin{proof}
            Assume $a, b \in \mathbb{Z}$ and $a \ne b$.
            Assume by way of contradiction, $ab = 1$.
            By our assumption, $a \ne 0$ and $b \ne 0$. Since $a, b \in \mathbb{Z}$, 
            \begin{gather*}
                a^2 > 0 \text{ and } b^2 > 0\\
                a^2 \ge 1 \text{ and } b^2 \ge 1\\
                a^2b^2 \ge b^2 \text{ and } b^2a^2 \ge a^2 \text{ but } ab = 1\\
                1 \ge b^2 \text{ and } 1 \ge a^2 \contradiction
            \end{gather*}
            So $ab \ne 1$ and thus this completes the proof by contradiction.
        \end{proof}
\end{enumerate}
\end{document}
