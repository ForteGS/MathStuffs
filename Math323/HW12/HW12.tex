\documentclass{article}
\usepackage{amsmath, amssymb, verbatim, amsthm}
\newcommand{\contradiction}{\Rightarrow\!\Leftarrow}
% I need a title
\author{Minh Bui}
\title{Math 323 HW12}

\theoremstyle{claim}
\newtheorem{claim}{Claim}
\newtheorem{theorem}{Theorem}[section]
\newtheorem{corollary}{Corollary}[theorem]
\newtheorem{lemma}[theorem]{Lemma}
\theoremstyle{definition}
\newtheorem{definition}{Definition}
\begin{document}
% Generates the title
\maketitle
\begin{enumerate}
    \item[Problem 7.10: ] Prove that $\forall n, m \in \mathbb{Z}$ if $nm$ is even, then either $n$ is even or $m$ is even.
    \begin{proof}
        Let $n, m \in \mathbb{N}$. We will prove the contrapositive statement: If $n$ is odd and $m$ is odd, then $nm$ is odd.\\
        Assume $n$ is odd and $m$ is odd. Respectively,
        \begin{gather*}
            \exists k \in \mathbb{Z} \text{ s.t } n = 2k + 1\\
            \exists l \in \mathbb{Z} \text{ s.t } m = 2l + 1
        \end{gather*}
        Consider $nm$,
        \begin{align*}
            nm & = (2k + 1)(2l + 1)\\
            & = 4kl + 2l + 2k + 1\\
            & = 2(2kl + l + k) + 1
        \end{align*}
        We know $2kl + l + k \in \mathbb{Z}$, so $nm$ is odd. This completes the proof by contraposition.
    \end{proof}
    \item[Problem 7.12: ] Let $S = \{ x \in \mathbb{R} \mid x^{-1} \in \mathbb{N} \}$. Prove that 0 is the infimum of $S$.
    \begin{proof}
        We will need to establish 2 claims.
        \begin{enumerate}
            \item[1.] If $s \in \mathbb{S}$, then $0 \le s$.\\
                Assume $s \in \mathbb{S}$, so $s = n^{-1} = \frac{1}{n}$ where $n \in \mathbb{N}$. Since $n \in \mathbb{N}$, we know $\frac{1}{n} > 0$. So $\frac{1}{n} \ge 0$. This means $0$ is a lower bound on set $S$.
            \item[2.] If $x \in \mathbb{R}$ and $x > 0$, then $\exists t \in S \text{ s.t } t < x$.\\
                Assume $x \in \mathbb{R}$ and $x > 0$. By the Archimedean principle, $\exists n \in \mathbb{N}$ so that $0 < \frac{1}{n} < x$. Let $t = \frac{1}{n}$ so we know $t \in S$.
        \end{enumerate}
        Thus $0$ is the infimum of the set $S = \{ x \in \mathbb{R} \mid x^{-1} \in \mathbb{N} \}$.
    \end{proof}
    \item[Problem 7.*:] Prove the following statement: Let $\epsilon \in \mathbb{R}$ and let $\epsilon > 0$. $\exists n \in \mathbb{N}$ such that $\frac{1}{n} < \epsilon$.
        \begin{proof}
            Assume $\epsilon \in \mathbb{R}$ and $\epsilon > 0$. Since $\epsilon \in \mathbb{R}$, $\frac{1}{\epsilon} \in \mathbb{R}$. By the Archimedean principle, $\exists m \in \mathbb{N} \text{ s.t } \frac{1}{\epsilon} < m$. Since $\epsilon > 0$, $1 < \epsilon m$. So $\frac{1}{m} < \epsilon$ since $m \in \mathbb{N}$. Let $n = m$ and we are done.
        \end{proof}
\end{enumerate}
\end{document}
