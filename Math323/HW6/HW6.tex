\documentclass{article}
\usepackage{amsmath, amssymb, verbatim, amsthm}
\newcommand{\contradiction}{\Rightarrow\!\Leftarrow}
% I need a title
\author{Minh Bui}
\title{Math 323 HW6}

\theoremstyle{claim}
\newtheorem{claim}{Claim}

\theoremstyle{definition}
\newtheorem{definition}{Definition}
\begin{document}
% Generates the title
\maketitle
\begin{enumerate}
    \item[Problem 4.6:] Prove that addition in $\mathbb{Q}$ is well-defined.
        \begin{proof}
            We will prove the following statement: If $\frac{m}{n}, \frac{p}{q}, \frac{r}{s}, \frac{t}{u} \in \mathbb{Q}$ such that $\frac{m}{n} = \frac{r}{s}$ and $\frac{p}{q} = \frac{t}{u}$, then $\frac{m}{n} + \frac{p}{q} = \frac{r}{s} + \frac{t}{u}$.\\
            Assume $\frac{m}{n}, \frac{p}{q}, \frac{r}{s}, \frac{t}{u} \in \mathbb{Q}$ such that $\frac{m}{n} = \frac{r}{s}$ and $\frac{p}{q} = \frac{t}{u}$. So $ms = rn$ and $pu = qt$. Consider $ms = rn$ when we multiply both sides by the natural number $qu$.
            \begin{gather}
                ms = rn\\
                ms(qu) = rn(qu)
            \end{gather}
            Similarly, consider $pu = qt$ when we multiply both sides by the natural number $sn$.
            \begin{gather}
                pu = qt\\
                pu(sn) = qt(sn)
            \end{gather}
            Revisiting equation $(2)$ again
            \begin{gather*}
                ms(qu) = rn(qu)\\
                ms(qu) + pu(sn) = rn(qu) + pu(sn)\\
                ms(qu) + pu(sn) = rn(qu) + qt(sn) \text{ by (4) }\\
                sumq + supn = nqru + nqst \text{ by commutativity and associativity }\\
                su(mq + pn) = nq(ru + st)\\
                \frac{(mq + pn)}{nq} = \frac{(ru + st)}{su}\\
                \frac{m}{n} + \frac{p}{q} = \frac{r}{s} + \frac{t}{u}
            \end{gather*}
            And so we have proved that addition in $\mathbb{Q}$ is well-defined.
        \end{proof}
    \item[Problem 4.12:] Prove that multiplication in $\mathbb{Q}$ is commutative.
        \begin{proof}
            We need to prove: If $\frac{m}{n}, \frac{p}{q} \in \mathbb{Q}$, then $\frac{m}{n} \cdot \frac{p}{q} = \frac{p}{q} \cdot \frac{m}{n}$.\\
            Assume $\frac{m}{n}, \frac{p}{q} \in \mathbb{Q}$, consider the expression
            \begin{align*}
                \frac{m}{n} \cdot \frac{p}{q} &= \frac{mp}{nq}\\
                &= \frac{pm}{nq} \text{ multiplication is commutative in } \mathbb{Z}\\
                &= \frac{pm}{qn} \text{ multiplication is commutative in } \mathbb{N}\\
                &= \frac{p}{q} \cdot \frac{m}{n}
            \end{align*}
            So multiplication is commutative in $\mathbb{Q}$.
        \end{proof}
    \item[Problem 4.17:] Prove that there is no $r \in \mathbb{Q}$ so that $r^2 = 6$.
        \begin{proof}
            Assume by way of contradiction, $\exists r \in \mathbb{Q}$ so that $r^2 = 6$. Since $r \in \mathbb{Q}$, $\exists i \in \mathbb{Z}$ and $\exists j \in \mathbb{N}$ so that $r = \frac{i}{j}$. Let $D$ be a set of natural numbers such that
            \begin{equation}
                D = \{k \in \mathbb{N} \mid \exists l \in \mathbb{Z} \text{ so that } r = \frac{l}{k}\}
            \end{equation}
            Since $r = \frac{i}{j}$, we know $j \in D$ and so $D \ne \emptyset$. By the Well Ordering principle, $D$ has a minimum. Call it $m_D$. So $\exists m_N \in \mathbb{Z}$ so that $r = \frac{m_N}{m_D}$ \\
            Consider $r = \frac{m_N}{m_D}$.
            \begin{gather*}
                r^2 = \frac{m_N^2}{m_D^2} = 6\\
                m_N^2 = 6m_D^2\\
                m_N^2 = 2.3.m_D^2
            \end{gather*}
            This means $m_N^2$ is even. And so is $m_N$, which means $\exists n \in \mathbb{Z}$ so that $m_N = 2n$.
            We have
            \begin{gather*}
                m_N^2 = 6m_D^2\\
                (2n)^2 = 6m_D^2\\
                4n^2 = 6m_D^2\\
                2n^2 = 3m_D^2
            \end{gather*}
            This means $3m_D^2$ is even. And so is $m_D^2$. And so is $m_D$, which means $\exists m \in \mathbb{Z}$ so that $m_D = 2m$.
            So now we can rewrite $r$ as $r = \frac{m_N}{m_D} = \frac{2n}{2m} = \frac{n}{m}$. And so $m \in D$.\\
            But our assumption says that $m_D$ is the minimum of $D$. And now we have $m_D = 2m$ and $m \in D$. This contradicts our assumption that $m_D$ is the minimum of $D$. $\contradiction$\\
            So $\nexists r \in \mathbb{Q}$ such that $r^2 = 6$. 
        \end{proof}
\end{enumerate}
\end{document}
