\documentclass{article}
\usepackage{amsmath, amssymb, verbatim, amsthm}
\newcommand{\contradiction}{\Rightarrow\!\Leftarrow}
% I need a title
\author{Minh Bui}
\title{Linear Algebra Chapter 1}

\theoremstyle{claim}
\newtheorem{claim}{Claim}
\newtheorem{theorem}{Theorem}[section]
\newtheorem{corollary}{Corollary}[theorem]
\newtheorem{lemma}[theorem]{Lemma}
\theoremstyle{definition}
\newtheorem{definition}{Definition}
\begin{document}
% Generates the title
\maketitle
\begin{enumerate}
    \item[Section 1.2]
        \begin{enumerate}
            \item[1.] Label the following statements as being true or false.
                \begin{enumerate}
                    \item Every vector space contains a zero vector. 
                    \item[] True by the property of vector space.
                    \item A vector space may have more than 1 zero vector.
                    \item[] False by the uniqueness of the additive identity.
                    \item In any vector space $ax = bx$ implies that $a = b$.
                    \item[] False. $ax - bx = 0$. So $x(a - b) = 0$. $x$ could be a zero vector while $a \ne b$.
                    \item In any vector space $ax = ay$ implies $x = y$.
                    \item[] False. What if $a = 0$?
                    \item An element of $F^n$ may be regarded as an element of $M_{n \times 1}(F)$.
                    \item[] True. Since n-tuple can be written as a column vector.
                    \item An $m \times n$ matrix has $m$ columns and $n$ rows.
                    \item[] False. An $m \times n$ matrix has $m$ rows and $n$ columns.
                    \item In the vector space $P(F)$ only polynomials of the same degree may be added.
                    \item[] False.
                    \item If $f$ and $g$ are polynomials of degree $n$, then $f + g$ is a polynomial of degree $n$.
                    \item[] False. Consider $x + (-x) = 0$
                    \item If $f$ is a polynomial of degree $n$ and $c$ is a nonzero scalar, then $cf$ is a polynomial of degree $n$.
                    \item[] True.
                    \item A nonzero element of $F$ may be considered to be an element of $P(F)$ having degree zero.
                    \item[] True.
                    \item Two funstions in $\mathcal{F}(S,F)$ are equal if and only if they have the same values at each element of $S$.
                    \item[] True.
                \end{enumerate}
            \item[2.] Write the zero vector of $M_{3 \times 4}(F)$.
            \item[] \emph{Solution. }\
                \begin{equation*}
                    \begin{pmatrix}
                        0 & 0 & 0 & 0\\
                        0 & 0 & 0 & 0\\
                        0 & 0 & 0 & 0\\
                    \end{pmatrix}
                \end{equation*}
            \item[3.] If
                \begin{equation*}
                    M =
                    \begin{pmatrix}
                        1 & 2 & 3\\
                        4 & 5 & 6
                    \end{pmatrix}
                \end{equation*}
                what are $M_{13}, M_{21}$, and $M_{22}$?
            \item[] \emph{Solution. } $M_{13} = 3$. $M_{21} = 4$. $M_{22} = 5$.
            \item[4.] Perform the indicated operations.
                \begin{enumerate}
                    \item[(a)]
                        $\begin{pmatrix}
                            2 & 5 & -3\\
                            1 & 0 & 7
                        \end{pmatrix} +
                        \begin{pmatrix}
                            4 & -2 & 5\\
                            -5 & 3 & 2
                        \end{pmatrix} =
                        \begin{pmatrix}
                            6 & 3 & 2\\
                            -4 & 3 & 9
                        \end{pmatrix}
                        $
                    \item[(b)]
                        $\begin{pmatrix}
                            -6 & 4\\
                            3 & -2\\
                            1 & 8
                        \end{pmatrix} + 
                        \begin{pmatrix}
                            7 & -5\\
                            0 & -3\\
                            2 & 0
                        \end{pmatrix} =
                        \begin{pmatrix}
                            1 & -1\\
                            3 & -5\\
                            3 & 8
                        \end{pmatrix}
                        $
                    \item[(c)]
                        $4\begin{pmatrix}
                            2 & 5 & -3\\
                            1 & 0 & 7
                        \end{pmatrix} =
                        \begin{pmatrix}
                            8 & 10 & -12\\
                            4 & 0 & 28
                        \end{pmatrix}
                        $
                    \item[(d)]
                        $-5\begin{pmatrix}
                            -6 & 4\\
                            3 & -2\\
                            1 & 8
                        \end{pmatrix} = 
                        \begin{pmatrix}
                            30 & -20\\
                            -15 & 10\\
                            -5 & -40
                        \end{pmatrix}
                        $
                    \item[(e)] $(2x^4 - 7x^3 + 4x + 3) + (8x^3 + 2x^2 - 6x + 7) = 2x^4 + x^3 + 2x^2 - 2x + 10$.
                    \item[(f)] $(-3x^3 + 7x^2 + 8x - 6) + (2x^3 - 8x + 10) = -x^3 + 7x^2 + 4$
                    \item[(g)] $5(2x^7 - 6x^4 + 8x^2 - 3x) = 10x^7 - 30x^4 + 40x^2 - 15x$
                    \item[(h)] $3(x^5 - 2x^3 + 4x + 2) = 3x^5 - 6x^3 + 12x + 6$.
                \end{enumerate}
            \item[7.] Let $S = \{0, 1\}$ and $F = R$. In $\mathcal{F}(S,R)$ show that $f = g$ and $f + g = h$, where $f(x) = 2x + 1, g(x) = 1 + 4x - 2x^2$, and $h(x) = 5^x + 1$.
            \item[] \emph{Solution. }
                To show $f = g$, we need to show: if $s \in S$, then $f(s) = g(s)$.
                \begin{enumerate}
                    \item[-] For $s = 0$, $f(0) = 1$ and $g(0) = 1$. So $f(s) = g(s)$ for $s = 0$.
                    \item[-] For $s = 1$, $f(1) = 3$ and $g(1) = 1 + 4 - 2 = 3$. So $f(s) = g(s)$ for $s = 1$.
                \end{enumerate}
                So $f = g$.\\
                To show $f + g = h$, we need to show: if $s \in S$, then $(f + g)(s) = h(s)$.
                \begin{enumerate}
                    \item[-] For $s = 0$, $(f + g)(s) = f(s) + g(s) = f(0) + g(0) = 2$. $h(s) = h(0) = 5^0 + 1 = 2$.
                    \item[-] For $s = 1$, $(f + g)(s) = f(s) + g(s) = f(1) + g(1) = 6$. $h(s) = h(1) = 5^1 + 1 = 6$.
                \end{enumerate}
                So $f + g = h$.
            \item[8.] In any vector space $\mathcal{V}$, show that $(a + b)(x + y) = ax + ay + bx + by$ for any $x, y \in \mathcal{V}$ and any $a, b \in F$.
            \item[] \emph{Solution. } Let $x, y \in \mathcal{V}$. Let $a, b \in F$. Consider
                \begin{gather*}
                    ax + ay = a(x + y) \text{ and } bx + by = b(x + y) \text{ by VS7 }\\
                    ax + ay + bx + by = a(x + y) + b(x + y)\\
                    ax + ay + bx + by = (a + b)(x + y) \text{ by VS8 }
                \end{gather*}
            \item[9.] Prove that
                \begin{enumerate}
                    \item[]\begin{corollary}
                        The vector $0 \in \mathcal{V}$ s.t $x + 0 = x$ for all $x \in \mathcal{V}$ is unique.
                    \end{corollary}
                    \begin{proof}
                        Let $\mathcal{V}$ be a vector space in $F$. Assume $x \in \mathcal{V}$. Assume by way of contradiction, there is a $\mathcal{O} \in \mathcal{V}$ s.t $x + \mathcal{O} = x$ and $\mathcal{O} \ne 0$. Since $\mathcal{V}$ is a vector space, $\exists 0 \in \mathcal{V}$ s.t $x + 0 = x$.\\
                        Since $\mathcal{O}$ is an additive identity in $\mathcal{V}$, $(x + 0) + \mathcal{O} = x + 0$. So $x + (0 + \mathcal{O}) = x + 0$ by associativity. By the Cancellation Law for vector addition, $0 + \mathcal{O} = 0$.\\
                        Since $0$ is an additive identity in $\mathcal{V}$, $(x + \mathcal{O}) + 0 = x + \mathcal{O}$. So $x + (0 + \mathcal{O}) = x + \mathcal{O}$ by associativity. By the Cancellation Law for vector addition, $0 + \mathcal{O} = \mathcal{O}$.\\
                        So $0 + \mathcal{O} = 0 = \mathcal{O}$ which contradicts our assumption of $\mathcal{O} \ne 0$. So $0 = \mathcal{O}$. Therefore the vector $0 \in \mathcal{V}$ is unique.
                    \end{proof}
                    \item[]\begin{corollary}
                        For all $x \in \mathcal{V}$, there is a vector $y \in \mathcal{V}$ so that $x + y = 0$. Prove that this vector $y$ is unique.
                    \end{corollary}
                    \begin{proof}
                        Pretty much the same approach as the previous corollary.
                    \end{proof}
                \end{enumerate}
            \item[10.] Let $\mathcal{V}$ denote the set of all differentiable real-valued functions defined on the real line. Prove that $\mathcal{V}$ is a vector space under the operation of addition and scalar multiplication defined by
                \begin{equation*}
                    (f + g)(s) = f(s) + g(s) \text{ and } (cf)(s) = c[f(s)]
                \end{equation*}
                \begin{proof}
                    Let $f, g$ be differentiable real-valued functions in $\mathcal{V}$. Since $(f + g)(s) = f(s) + g(s)$ is differentiable on real numbers, addition is closed in $\mathcal{V}$. Since $(cf)(s) = c[f(s)]$ for some $c \in \mathbb{R}$ is differentiable on real numbers, scalar multiplication is closed in $\mathcal{V}$. And the function $f = 0$ would be the zero vector in the vector space $\mathcal{V}$. Thus, $\mathcal{V}$ is a vector space.
                \end{proof}
            \item[11.] Let $\mathcal{V} = \{0\}$ consist of a single vector $0$ and define $0 + 0 = 0$ and $c0 = 0$ for each scalar $c \in \mathbb{F}$. Prove that $\mathcal{V}$ is a vector space over $\mathbb{F}$. Here $\mathcal{V}$ is called the \emph{zero vector space}.
            \begin{proof}
                Since $0$ is the only element of $\mathcal{V}$, all conditions and properties for vector space $\mathcal{V}$ can be easily checked.
            \end{proof}

        
        \end{enumerate}
\end{enumerate}
\end{document}
