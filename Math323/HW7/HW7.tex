\documentclass{article}
\usepackage{amsmath, amssymb, verbatim, amsthm}
\newcommand{\contradiction}{\Rightarrow\!\Leftarrow}
% I need a title
\author{Minh Bui}
\title{Math 323 HW7}

\theoremstyle{claim}
\newtheorem{claim}{Claim}
\newtheorem{theorem}{Theorem}[section]
\newtheorem{corollary}{Corollary}[theorem]
\newtheorem{lemma}[theorem]{Lemma}
\theoremstyle{definition}
\newtheorem{definition}{Definition}
\begin{document}
% Generates the title
\maketitle
\begin{enumerate}
    \item[Problem 5.2:] Let $\mathbb{F}$ be an ordered field.
        \begin{enumerate}
            \item Prove: For all $n \in \mathbb{N}$, if $a_1 \in \mathbb{F}; a_2 \in \mathbb{F}; ... a_n \in \mathbb{F}$, then $\sum\limits_{k=1}^n a_k^2 \ge 0$.
            \item Prove: For all $n \in \mathbb{N}$, if $a_1 \in \mathbb{F}; a_2 \in \mathbb{F}; ... a_n \in \mathbb{F}$, and $\sum\limits_{k=1}^n a_k^2 = 0$, then $a_1 = a_2 = a_3 = ... = a_n = 0$.
        \end{enumerate}
    \item[] \emph{Solution.}
        \begin{enumerate}
            \item We will prove (a) using induction on $n$.
                \begin{proof}
                    Assume $n \in \mathbb{N}$. We need to prove two claims.
                    \begin{enumerate}
                        \item We want to show that: If for $n = 1$, then if $a_1 \in \mathbb{F}; ... ;a_n \in \mathbb{F}$, then $\sum\limits_{k=1}^n a_k^2 \ge 0$.\\
                        Assume $n = 1$. Assume $a_1 \in \mathbb{F}; ...; a_n \in \mathbb{F}$. So $a_1 \in \mathbb{F}$. Consider the expression 
                        \begin{equation*}
                            \sum\limits_{k=1}^n a_k^2 = \sum\limits_{k=1}^1 a_k^2 = a_1^2 \ge 0
                        \end{equation*}
                        So for $n = 1$, if $a_1,...,a_n \in \mathbb{F}$, then $\sum\limits_{k=1}^n a_k^2 \ge 0$.
                    \item We also want to show that: If for $n = n_0$, if $a_1 \in \mathbb{F}; ... ; a_n \in \mathbb{F}$, then $\sum\limits_{k=1}^n a_k^2 \ge 0$ then for $n = n_0 + 1$, if $a_1, ..., a_n \in \mathbb{F}$, then $\sum\limits_{k=1}^n a_k^2 \ge 0$.\\
                        Assume for $n = n_0$. Assume if $a_1, ..., a_n \in \mathbb{F}$, then  $\sum\limits_{k=1}^n a_k^2$. And so if $a_1, ..., a_{n_0} \in \mathbb{F}$, then $\sum\limits_{k=1}^{n_0} a_k^2 \ge 0$. Assume $\exists a_{n_0 + 1} \in \mathbb{F}$.\\ 
                        We know $\sum\limits_{k=1}^{n_0} a_k^2 \ge 0$ by our inductive hypothesis. We also know $a_{n_0 + 1}^2 \ge 0$. So we can conclude that
                        \begin{equation*}
                            \sum\limits_{k=1}^{n_0} a_k^2 + a_{n_0 + 1}^2 = \sum\limits_{k=1}^{n_0 + 1} a_k^2 \ge 0
                        \end{equation*}
                       So we have proved that: If for $n = n_0$, if $a_1 \in \mathbb{F}; ... ; a_n \in \mathbb{F}$, then $\sum\limits_{k=1}^n a_k^2 \ge 0$ then for $n = n_0 + 1$, if $a_1, ..., a_n \in \mathbb{F}$, then $\sum\limits_{k=1}^n a_k^2 \ge 0$.\\  
                    \end{enumerate}
                    Proving (i) and (ii) thus completes our proof by induction on $n$.
                \end{proof}
            \item We will prove (b) using induction on $n$.
                \begin{proof}
                    Assume $n \in \mathbb{N}$. We need to prove two claims.
                    \begin{enumerate}
                        \item We want to show that: If for $n = 1$, then if $a_1, ..., a_n \in \mathbb{F}$ and $\sum\limits_{k=1}^n a_k^2 = 0$, then $a_1 = a_2 = ... = a_n = 0$.\\
                            Assume $n = 1$. Assume $a_1, ..., a_n \in \mathbb{F}$. So $a_1 \in \mathbb{F}$. Assume $\sum\limits_{k=1}^n a_k^2 = 0$, meaning
                            \begin{equation*}
                                \sum\limits_{k=1}^n a_k^2 = \sum\limits_{k=1}^1 a_k^2 = a_1^2 = 0\\
                            \end{equation*}
                            So $a_1^2 = 0$. By trichotomy, we can imply that $a_1 = 0$.\\
                            So we have proved that for $n = 1$, then if $a_1, ..., a_n \in \mathbb{F}$ and $\sum\limits_{k=1}^n a_k^2 = 0$, then $a_1 = a_2 = ... = a_n = 0$.\\
                        \item We also want to show that: If for $n = n_0$, if $a_1, ..., a_n \in \mathbb{F}$ and $\sum\limits_{k=1}^n a_k^2 = 0$, then $a_1 = a_2 = ... = a_n = 0$, then for $n = n_0 + 1$, if $a_1, ..., a_n \in \mathbb{F}$ and $\sum\limits_{k=1}^n a_k^2 = 0$, then $a_1 = a_2 = ... = a_n = 0$.\\
                            Assume for $n = n_0$, $a_1, ..., a_n \in \mathbb{F}$. Assume $\sum\limits_{k=1}^n a_k^2 = 0$. By our inductive hypothesis, we have $a_1 = a_2 = ... = a_{n_0} = 0$. Assume $\exists a_{n_0 + 1} \in \mathbb{F}$. By trichotomy, exactly one of these holds: $a_{n_0 + 1} > 0, a_{n_0 + 1} = 0$, or $a_{n_0 + 1} < 0$. Whatever case that is, we know $a_{n_0 + 1}^2 \ge 0$. Now, we consider $\sum\limits_{k=1}^{n_0} a_k^2$.
                            \begin{equation}
                                \sum\limits_{k=1}^{n_0} a_k^2 + a_{n_0+1}^2 = \sum\limits_{k=1}^{n_0 + 1} a_k^2 \ge 0\\
                            \end{equation}
                    Our inductive hypothesis says that $\sum\limits_{k=1}^{n_0} a_k^2 = 0$.\\
                    And we know $a_{n_0 + 1}^2 \ge 0$. So the equality of the equation (1) holds if and only if $a_{n_0 + 1}^2 = 0$. This means that it must be the case that $a_{n_0 + 1} = 0$.\\
                    And so $a_1 = a_2 = ... = a_{n_0} = a_{n_0 + 1} = 0.$\\
                    So we have proved that:  If for $n = 1$, then if $a_1, ..., a_n \in \mathbb{F}$ and $\sum\limits_{k=1}^n a_k^2 = 0$, then $a_1 = a_2 = ... = a_n = 0$.\\
                    \end{enumerate}
                    Proving both claim (i) and (ii) completes our proof by induction on $n$. 
                \end{proof}
        \end{enumerate}
    \item[Problem 5.7:] Let $\mathbb{F}$ be any ordered field with $a, b, c \in \mathbb{F}$, prove that $\vert a - c \vert \le \vert a - b \vert + \vert c - b \vert$.
        \begin{proof}
            Assume $a, b, c \in \mathbb{F}$. To prove the above statement, we need to prove a lemma.
            \begin{lemma}
                Let $\mathbb{F}$ be any ordered field with $a, b \in \mathbb{F}$: $|a - b| = |b - a|$.
                \begin{proof}
                    Assume $a, b \in \mathbb{F}$. By the definition of absolute value we have
                    \begin{equation*}
                        |a - b| = \left\{
                            \begin{array}{rl}
                                - (a - b) = b - a & \text{if } (a - b) < 0 \Leftrightarrow a < b,\\
                                0 & \text{if } a - b = 0 \Leftrightarrow a = b,\\
                                (a - b) & \text{if } a - b > 0 \Leftrightarrow a > b.
                            \end{array} \right.
                    \end{equation*}
                    And
                    \begin{equation*}
                        |b - a| = \left\{
                            \begin{array}{rl}
                                - (b - a) = a - b & \text{if } (b - a) < 0 \Leftrightarrow a > b,\\
                                0 & \text{if } b - a = 0 \Leftrightarrow a = b,\\
                                (b - a) & \text{if } b - a > 0 \Leftrightarrow a < b.
                            \end{array} \right.
                    \end{equation*}
                    So we can conclude that $|a - b| = |b - a|$ for $a, b \in \mathbb{F}$.
                \end{proof}
            \end{lemma}
            Now we are ready to prove our main statement. Consider $|a - b| + |c - b|$
            \begin{gather*}
                |a - b| + |c - b| = |a - b| + |b - c| \text{ by lemma 0.1}\\
                |a - b| + |b - c| \ge | (a - b) + (b - c) | \text{ by the Triangle Inequality} \\
                |a - b| + |b - c| \ge | a - b + b - c |\\
                |a - b| + |b - c| \ge | a - c|\\
                \text{So } |a - b| + |c - b| \ge | a - c |
            \end{gather*}
        \end{proof}
    \item[Problem 5.8a:] Let $\mathbb{F}$ be any ordered field. Prove that if $a, b \in \mathbb{F}$ so that $a < b$, then $\forall n \in \mathbb{N}$, there are numbers $x_i \in \mathbb{F}$ so that $a < x_1 < x_2 < ... < x_n < b$.
        \begin{proof}
            Assume $a, b \in \mathbb{F}$ so that $a < b$. Since we have $n \in \mathbb{N}$, we can attempt to prove the above proposition using a proof by induction on $n$. We need to prove two claims.
            \begin{enumerate}
                \item[]{(i)} We want to prove that: If for $n = 1$, then $a < x_1 < x_2 < ... < x_n < b$.\\
                    Assume $n = 1$. Since $a, b \in \mathbb{F}$ and $a < b$, by the Average theorem, $\exists x_1 \in \mathbb{F}$ such that $a < x_1 < b$. And so for $n = 1$, $a < x_1 < x_2 < .. < x_n < b$.
                \item[]{(ii)} We also want to prove: If for $n = n_0$, $a < x_1 < x_2 < ... < x_n < b$, then for $n = n_0 + 1$, $a < x_1 < x_2 < ... < x_n < b$.\\
                    Assume for $n = n_0$, $a < x_1 < x_2 < ... < x_n < b$, meaning
                    \begin{equation}
                        a < x_1 < x_2 < ... < x_{n_0 - 1} < x_{n_0} < b \text{ where } x_1, x_2, ..., x_{n_0} \in \mathbb{F}
                    \end{equation}
                    From (1) we know $x_{n_0} < b$. Again, by the Average theorem, $\exists x_{n_0 + 1} \in \mathbb{F}$ such that $x_{n_0} < x_{n_0 + 1} < b$.
                    In conclusion, we know
                    \begin{equation*}
                        a < x_1 < x_2 < ... < x_{n_0 - 1} < x_{n_0} < x_{n_0 + 1} < b,
                    \end{equation*}
                    which is what we want to prove.\\
            \end{enumerate}
            Proving both claim (i) and (ii) thus completes our proof by induction on $n$.
        \end{proof}
\end{enumerate}
\end{document}
