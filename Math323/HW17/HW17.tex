\documentclass{article}
\usepackage{amsmath, amssymb, verbatim, amsthm}
\newcommand{\contradiction}{\Rightarrow\!\Leftarrow}
% I need a title
\author{Minh Bui}
\title{Math 323 HW17}

\theoremstyle{claim}
\newtheorem{claim}{Claim}
\newtheorem{theorem}{Theorem}[section]
\newtheorem{corollary}{Corollary}[theorem]
\newtheorem{lemma}[theorem]{Lemma}
\theoremstyle{definition}
\newtheorem{definition}{Definition}
\begin{document}
% Generates the title
\maketitle
\begin{enumerate}
    \item[Problem 11.8:] Define $f: [3, \infty) \rightarrow [-6, \infty)$ by $f(x) = x^2 - 6x + 3$.
        \begin{enumerate}
            \item[(a)] Find a function $g: [-6, \infty) \rightarrow [3, \infty)$ so that $(g \circ f)(x) = x$ and $(f \circ g)(x) = x$.
            \item[] \emph{Solution. } $g(x) = 3 + \sqrt{6 + x}$ or $g(x) = 3 - \sqrt{6 + x}$.
            \item[(b)] Prove that $f(x)$ is bijective.
            \begin{proof}
                We will need to prove 2 claims.
                \begin{enumerate}
                    \item[1.] $f(x)$ is injective.\\
                        Precisely, we want to show that: if $x_1, x_2 \in [3; \infty)$ and $f(x_1) = f(x_2)$, then $x_1 = x_2$.\\
                        Assume $x_1, x_2 \in [3; \infty)$ and $f(x_1) = f(x_2)$.
                        So then $f(x_1) = x_1^2 - 6x_1 + 3$ and $f(x_2) = x_2^2 + 6 - 6x_2 + 3$. Because $f(x_1) = f(x_2)$, $x_1^2 - 6x_1 + 3 = x_2^2 - 6x_2 + 3$. So
                        \begin{gather*}
                            x_1^2 - 6x_1 = x_2^2 - 6x_2\\
                            x_1^2 - x_2^2 - 6x_1 + 6x_2 = 0\\
                            (x_1 + x_2)(x_1 - x_2) - 6(x_1 - x_2) = 0\\
                            (x_1 - x_2)(x_1 + x_2 + 6) = 0\\
                        \end{gather*}
                        Since $x_1, x_2 \in [3; \infty)$, $x_1 + x_2 + 6 > 0$, so then $x_1 - x_2 = 0$. Thus $x_1 = x_2$.
                    \item[2.] $f(x)$ is surjective.\\
                        We want to show that: if $b \in [-6, \infty)$, then $\exists a \in [3, \infty) \text{ s.t } b = a^2 - 6a + 3$.\\
                        Assume $b \in [-6, \infty)$. Let $a = 3 \pm \sqrt{6 + b}$. We know $\sqrt{6 + b} \in \mathbb{R}$ for $b \in [-6, \infty)$. So $3 \pm \sqrt{6 + b} \in \mathbb{R}$ for $b \in [-6, \infty)$. We also observe that $a = 3 \pm \sqrt{6 + b} \in [3, \infty)$ for $b \in [-6, \infty)$. Thus $f(x)$ is surjective on $[3, \infty) \rightarrow [-6, \infty)$.
                \end{enumerate}
                Since $f(x)$ is both injective and surjective, $f(x)$ is bijective.
            \end{proof}
        \end{enumerate}
    \item[Problem 11.9:] Let $f: \mathbb{R} \rightarrow \mathbb{R}$ be given by $f(x) = x^3 + 5x - 8$. Prove that $f(x)$ is injective. (Hint: Can't use calculus, prove that $f(a) - f(b) = (a - b) \cdot g(a, b)$)
        \begin{proof}
            Let $f: \mathbb{R} \rightarrow \mathbb{R}$ be given by $f(x) = x^3 + 5x - 8$. We want to show that $f(x)$ is injective. Precisely: if $a_1, a_2 \in \mathbb{R}$ and $f(a_1) = f(a_2)$, then $a_1 = a_2$.\\
            Assume $a_1, a_2 \in \mathbb{R}$ and $f(a_1) = f(a_2)$. So then
            \begin{gather*}
                a_1^3 + 5a_1 - 8 = a_2^3 + 5a_2 - 8\\
                a_1^3 + 5a_1 - a_2^3 - 5a_2 = 0\\
                (a_1^3 - a_2^3) + 5(a_1 - a_2) = 0\\
                ( (a_1 - a_2)^3 + 3a_1a_2(a_1 - a_2)) + 5(a_1 - a_2) = 0\\
                (a_1 - a_2)((a_1 - a_2)^2 + 3a_1a_2 + 5) = 0
            \end{gather*}
            We need to consider 2 cases:
            \begin{itemize}
                \item $a_1 - a_2 = 0$.\\ Assume that is the case, so then $a_1 = a_2$ and we are done.
                \item $(a_2 - a_2)^2 + 3a_1a_2 + 5 = 0$.\\
                    Assume BWOC, $(a_1 - a_2)^2 + 3a_1a_2 + 5 = 0$.
                    We know $(a_1 - a_2)^2 \ge 0$. But then $(a_1 - a_2)^2 = -3a_1a_2 - 5$ and $-3a_1a_2 - 5 \ge 0$. The equality only holds when $a_1 = a_2 = 0$. So then $-3a_1a_2 - 5 = -5 \ge 0$, which is a contradiction. So $(a_1 - a_2)^2 + 3a_1a_2 + 5 \ne 0$.
            \end{itemize}
        \end{proof}
    \item[Problem 12.3:] Let $f: \mathbb{R} \rightarrow \mathbb{R}$ given by $f(x) = \frac{1}{x^2}$. Find the following.
        \begin{enumerate}
            \item $f((0, 3)) = (\frac{1}{9}, \infty)$
            \item $f([0, 4) = [\frac{1}{16}, \infty)$
            \item $f(\mathbb{R}) = (0, \infty)$
            \item $f([-2, 3]) = [\frac{1}{9}, \infty)$
            \item $f(\emptyset) = \emptyset$
            \item $f^{-1}((0, \infty)) = \mathbb{R} \setminus \{ 0 \}$
            \item $f^{-1}((-1, 1)) = [1, \infty) \cup (-\infty, -1]$
            \item $f^{-1}((\frac{1}{4}, 1]) = [1, 2) \cup (-2, -1]$
            \item $f^{-1}(\mathbb{R}) = \mathbb{R} \setminus \{ 0 \}$
            \item $f^{-1}(\emptyset) = \emptyset$
        \end{enumerate}
    \item[Problem 12.4:] Let $f: \mathbb{R} \rightarrow \mathbb{R}$ given by $f(x) = x^3 - x$. Find the following.
        \begin{enumerate}
            \item $f(\{-1, 0, 1\}) = \{ 0 \}$
            \item $f([-1, 1]) = [\frac{-2\sqrt{3}}{9}, \frac{2\sqrt{3}}{9}]$
            \item $f((-1, 1)) = [\frac{-2\sqrt{3}}{9}, \frac{2\sqrt{3}}{9}]$
            \item $f((-5, 5)) = (-120, 120)$
            \item $f(\mathbb{R}) = \mathbb{R}$
            \item $f(\emptyset) = \emptyset$
            \item $f^{-1}(\{ 0 \}) = \{-1, 0 , 1\}$
            \item $f^{-1}((0, \infty)) = (1, \infty) \cup (-1, 0)$
            \item $f^{-1}((-120, 120)) = (-5, 5)$
            \item $f^{-1}(\mathbb{R}) = \mathbb{R}$
            \item $f^{-1}(\emptyset) = \emptyset$
        \end{enumerate}
\end{enumerate}
\end{document}
