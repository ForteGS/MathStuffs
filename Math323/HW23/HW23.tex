\documentclass{article}
\usepackage{amsmath, amssymb, verbatim, amsthm}
\newcommand{\contradiction}{\Rightarrow\!\Leftarrow}
% I need a title
\author{Minh Bui}
\title{Math 323 HW23}

\theoremstyle{claim}
\newtheorem{claim}{Claim}
\newtheorem{theorem}{Theorem}[section]
\newtheorem{corollary}{Corollary}[theorem]
\newtheorem{lemma}[theorem]{Lemma}
\theoremstyle{definition}
\newtheorem{definition}{Definition}
\begin{document}
% Generates the title
\maketitle
\begin{enumerate}
    \item[Problem 16.1:] Prove that for all $A \subseteq \mathbb{R}$:
        \begin{enumerate}
            \item $Int(A) \subseteq A$
                \begin{proof}
                    Assume $x \in Int(A)$. So then $\exists \epsilon > 0$ so that $N(x, \epsilon) \subseteq A$. We know $x \in N(x, \epsilon)$. But $N(x, \epsilon) \subseteq A$. So then $x \in A$. Thus $Int(A) \subseteq A$.
                \end{proof}
            \item $A \subseteq Cl(A)$
                \begin{proof}
                    Assume $x \in A$. Want to show $x \in Cl(A)$, which means $x \in Int(A) \cup \partial(A)$. By (a), we know if $a \in Int(A)$, then $a \in A$. So then we have 2 cases:
                    \begin{enumerate}
                        \item[i.] $x \in Int(A)$. Then $x \in Int(A) \cup \partial(A)$ and thus $x \in Cl(A)$.
                        \item[ii.] $x \notin Int(A)$. In short, $x \in A$ and $x \notin Int(A)$.
                    \end{enumerate}
                \end{proof}
            \item $A^{\circ} \subseteq A$
                \begin{proof}
                    Assume $x \in A^{\circ}$. So then $\exists \epsilon > 0$ so that $N(x, \epsilon) \cap A = \{x\}$. So then $x \in A$. Thus $A^{\circ} \subseteq A$.
                \end{proof}
            \item $A^{\circ} \subseteq \partial(A)$
                \begin{proof}
                    Assume $a \in A^{\circ}$. So $\exists \epsilon > 0$ so that $N(a, \epsilon) \cap A = \{ a \}$. So then $N(a, \epsilon) \cap A \ne \emptyset$. By the definition of $\epsilon$-neighborhood, $N(a, \epsilon) = \{ x \in \mathbb{R} \mid |x - a| < \epsilon \}$. We also know $N(x, \epsilon) \cap \mathbb{R} \setminus A \ne \emptyset$.So then $a \in \partial(A)$.
                \end{proof}
            \item $Int(A) \subseteq A'$.
        \end{enumerate}
    \item[Problem 16.4:] Let $A \subseteq \mathbb{R}$ with $A \ne \emptyset$. Prove that if $A$ is bounded below, then $Inf(A) \in \partial(A)$.
        \begin{proof}
            Let $A \subseteq \mathbb{R}$ with $A \ne \emptyset$. Assume $A$ is bounded below. By the completeness axiom, $A$ has an infimum, denoted by $Inf(A)$. $Inf(A)$ has the following properties:
            \begin{enumerate}
                \item[1.] If $a \in A$, then $Inf(A) \le a$.
                \item[2.] If $x \in \mathbb{R}$ and $x > Inf(A)$, then $\exists \alpha \in A$ s.t $ \alpha < x$.
            \end{enumerate}
            We want to prove that $ \forall \epsilon > 0, N(Inf(A), \epsilon) \cap (\mathbb{R} \setminus A) \ne \emptyset$ and $N(Inf(A), \epsilon) \cap A \ne \emptyset$. \\
            Let $\epsilon > 0$. We know $Inf(A) - \epsilon \notin A$. So then $(Inf(A) - \epsilon, Inf(A) + \epsilon) \cap (\mathbb{R} \setminus A) \ne \emptyset$. This means $N(Inf(A), \epsilon) \cap (\mathbb{R} \setminus A) \ne \emptyset$.\\
            Let $x = Inf(A) + \epsilon$. By (2), $\exists \alpha \in A$ s.t $ \alpha < Inf(A) + \epsilon$.So then $\alpha \in (Inf(A) - \epsilon, Inf(A) + \epsilon)$, which means $\alpha \in N(Inf(A), \epsilon)$.Thus $N(Inf(A), \epsilon) \cap A \ne \emptyset$.
        \end{proof}
    \item[Problem 16.6a:] Let $A \subseteq B \subseteq \mathbb{R}$. Prove the following: $Int(A) \subseteq Int(B)$.
        \begin{proof}
            Let $A \subseteq B \subseteq \mathbb{R}$. Assume $x \in Int(A)$. So then $\exists \epsilon > 0$ such that $N(x, \epsilon) \subseteq A$. But we know $N(x, \epsilon) = \{ r \in \mathbb{R} \mid |r - x| < \epsilon\}$, which means $x \in N(x, \epsilon)$. Since $N(x, \epsilon) \subseteq A$ and we also know $A \subseteq B$, $N(x, \epsilon) \subseteq B$. This means $x \in Int(B)$.
        \end{proof}
    \item[Problem 16.7a:] Let $A \subseteq B \subseteq \mathbb{R}$. Why can't we use this to prove the following results? $\partial(A) \subseteq \partial(B)$.
    \item[] \emph{Attempted answer.} Let $A \subseteq B \subseteq \mathbb{R}$. Let $x \in \partial(A)$ and let $\epsilon > 0$. So then $N(x, \epsilon) \cap A \ne \emptyset$ and $N(x, \epsilon) \cap (\mathbb{R} \setminus A) \ne \emptyset$. Since $A \subseteq B$ and $N(x, \epsilon) \cap A \ne \emptyset$, $N(x, \epsilon) \cap B \ne \emptyset$. Knowing $N(x, \epsilon) \cap (\mathbb{R} \setminus A) \ne \emptyset$ is not enough to say $N(x, \epsilon) \cap (\mathbb{R} \setminus B) \ne \emptyset$.
    \item[] \emph{Example. } Consider set $A = (1, 2)$ and $B = (0, 3)$. It's true that $A \subseteq B \subseteq \mathbb{R}$. But $\partial(A) = \{ 1, 2 \}$ and $\partial(B) = \{0, 3\}$ and $\partial(A) \nsubseteq \partial(B)$.
\end{enumerate}
\end{document}
